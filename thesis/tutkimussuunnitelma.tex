\documentclass[12pt,a4paper,finnish,oneside]{article}

% Valitse 'input encoding':
%\usepackage[latin1]{inputenc} % merkistökoodaus, jos ISO-LATIN-1:tä.
\usepackage[utf8]{inputenc}   % merkistökoodaus, jos käytetään UTF8:a
% Valitse 'output/font encoding':
%\usepackage[T1]{fontenc}      % korjaa ääkkösten tavutusta, bittikarttana
\usepackage{ae,aecompl}       % ed. lis. vektorigrafiikkana bittikartan sijasta
% Kieli- ja tavutuspaketit:
\usepackage[finnish]{babel}
% Muita paketteja:
% \usepackage{amsmath}   % matematiikkaa
\usepackage{url}       % \url{...}

% Kappaleiden erottaminen ja sisennys
\parskip 1ex
\parindent 0pt
\evensidemargin 0mm
\oddsidemargin 0mm
\textwidth 159.2mm
\topmargin 0mm
\headheight 0mm
\headsep 0mm
\textheight 246.2mm

\pagestyle{plain}

% ---------------------------------------------------------------------

\begin{document}

% Otsikkotiedot: muokkaa tähän omat tietosi

\title{TIK.kand research plan:\\[5mm]
Ad-hoc social interaction for sports}

\author{Ville Tainio\\
Aalto University\\
\url{ville.tainio@aalto.fi}}

\date{\today}

\maketitle

% ---------------------------------------------------------------------

% MUOKKAA TÄHÄN. Jos tarvitset tähän viitteitä, käytä
% tässä dokumentissa numeroviitejärjestelmää komennolla \cite{kahva}.
%
% Paljon kandidaatintöitä ohjanneen Vesa Hirvisalon tarjoama 
% sabluuna. Kursivoidut osat \emph{...} ovat kurssin henkilökunnan
% lisäämiä. 

\textbf{Kandidaatintyön nimi:} Ad-hoc social interaction for sports

\textbf{Työn tekijä:} Ville Tainio

\textbf{Ohjaaja:} David McGookin


\section{Abstract}

Tutkimusaiheen lyhyt kuvaus. Esittele aiheesi tiivistetysti.

This thesis covers the use of proximity technologies to create social
interaction between so called familiar strangers. The goal is to create a
prototype application that suggests social interaction between people that
come accross each other often while doing sports.

\section{Goals}

Mitä haluat saada selville? Mitkä ovat keskeisiä kysymyksiä? Mistä
näkökulmasta asiaa tarkastellaan?

Tutkimuskysymystä kannattaa siis rajata ja tarkentaa sekä huomioida
näkökulman merkitys. Ts. jänikset eläintieteen kannalta ovat eri aihe
kuin jänikset metsästyksen näkökulmasta.

I'm interested in finding out are people willing to get to know strangers
and start doing sports activities with them. A question also arises about how
willing people are to share their information to strangers who they come accross
often. The ultimate goal is to find out, whether it is possible to create new
relationships with the prototype application I am going to build.

\section{Material}

Millaisen aineiston varaan perustat tutkimuksesi? Arvioi materiaalin
riittävyyttä asetettuihin tavoitteisiin nähden.

Pitää olla siis hieman kuvaa siitä, minkälaisen materiaalin kanssa
ollaan tekemisissä ja mitä sellaisen käsittelyyn tarvitaan (etenkin
siis tarvittavan ajan puolesta; ts. kuinka monta tuntia/minuuttia per
lähde?).

My thesis will base on a literature review of similar projects. Similar projects
include creating ad-hoc social networks with proximity technologies
and different kinds of location aware applications for sports. In addition
to the literature review, I will perform a series of interviews to possible
users of the application. Usage data of the application will also benefit
this research, if enough data available.

\section{Methods}

Miten sen keräät materiaalisi tai saat sen käsiisi? Kuinka käsittelet
sen? Kuinka siitä tulee raportti?

Tavallaisesti kirjallisuustutkimuksen yhteydessä tämä on:
(a) lähderyhmien valinta,
(b) viitteiden ja lähteiden haku,
(c) lähteiden arviointi,
(d) lähteiden lukeminen,
(e) tiedon organisointi,
(f) raportointi.  % (f) tärkeää ettei jää vain lukemiseksi!

Kirjallisuustutkimuksen yleinen menetelmä pitää sovittaa tähän
nimenomaiseen aiheeseen sekä tekijän lähtökohtiin. Kuinka sinä teet
muistiinpanot (että myös kirjoitat etkä pelkästään lue). Eli tälle
pitää hieman miettiä omakohtaista vaiheistusta. Siis nähdä ihan
oikeasti, kuinka sinä saat tutkielman tehtyä.

Ja... raportointi ei ole kirjoittamista vaan jo kirjoitettujen
muistiinpanojen koostamista yhteneväksi teokseksi.

I will go through material related to my thesis subject and highlight
lines that are relevant to my topic. Afther that I will follow references
that are presented in those lines and read if there is anything relevant
in the referenced research paper. That will be the basis for the literature
review for my thesis.

For the interviews, I will create a set of questions that I will present to
every interviewee. I will present that data with number or quotes, based on
what is relevant to the current subject.

Creating the prototype application will result in a few steps. Firstly,
I will design the application based on methods that are proven in other research
papers and my interviews. Secondly, I will code the application. Lastly, I will
write about the implementation of each stage in the process to my thesis.

If I have time to gather any data related to the usage of the application,
I will present them and draw conclusions based on the data.


\section{Challenges}

Yleensä kaikkiin töihin liittyy kompastuskiviä. Ne on syytä tiedostaa
etukäteen. Yhdessä työssä aihe on suurpiirteinen (työn rajaaminen
vaikeaa), toisessa materiaalia on niukasti saatavissa, kolmannessa
taas materiaalia on hukkumiseen asti.  Eli, nämä pitäisi kyseisen
tutkimuksen osalta kirjata ylös, ja nähdä ne myös mahdollisuuksina
(positiivisina haasteina) ei ainostaan esteinä.

It's hard to estime how long creating the prototype application will take.
There is always a lot of uncertainty when creating software. Therefore, it's
a risk that my thesis will fall under the time required to create the prototype
application. It's going to be hard to design the application so that I have
time to create it and cut features that aren't necessary for creating a
successful research project.

\section{Resources}

Kuka tätä työtä tekee, kuka ohjaa, jne. Paljonko on käyttää
aikaa. Tarvitaanko muuta? (Onko työssä joku kokeellinen osuus?)

David McGookin will guide this project. We agreed to meet weekly
with David an review the progress I have made for the thesis.
In addition to meeting weekly, David will provide good resources
for conducting my research.

I will also come up with questions for an interview and conduct
hopefully around five or six interviews. The interview will consist
of multiple open quetions that try to figure out, whether people
are interested in this kind of interaction with strangers.

\section{Schedule}

Laadi tutkimustyölle ja raportoinnille realistinen aikataulu.
Huolehdi, että suunnitelmasi vastaa kandidaatin tutkielman sekä
seminaarin aikataulua sekä laajuutta.  \emph{Kurssiesitteessä omalle
  kirjoitusprosessille on arvioitu noin 6 op eli 160 tuntia eli noin 4
  viikkoa työtä.}

%\begin{tabular}{|p{30mm}|p{120mm}|}
%\hline
%abcd   & qwerty qwerty \\ \hline
%abcd   & qwerty qwerty \\ \hline
%abcd   & qwerty qwerty \\ \hline
%\end{tabular}


\section{Presentation}

Laadi lyhyt sisällysluettelo, jossa on hahmoteltuna kandityön pää- ja
alaluvut. Yleensä perusrunko on
(1) Johdanto,
(2) Tausta,
(3) Sisäluvut,
(4) Yhteenveto.
%
Sinun täytyy suunnitella oma raportointisi tähän sopivaksi. 

\emph{Rakenne tarkentuu työn edetessä. Tutkimussuunnitelmaan ei välttämättä tarvita lähdeluetteloa, mutta halutessasi voit sisällyttää tärkeimmät lähteet.}


% ---------------------------------------------------------------------
%
% ÄLÄ MUUTA MITÄÄN TÄÄLTÄ LOPUSTA

% Tässä on käytetty siis numeroviittausjärjestelmää. 
% Toinen hyvin yleinen malli on nimi-vuosi-viittaus.

% \bibliographystyle{plainnat}
\bibliographystyle{finplain}  % suomi
%\bibliographystyle{plain}    % englanti
% Lisää mm. http://amath.colorado.edu/documentation/LaTeX/reference/faq/bibstyles.pdf

% Muutetaan otsikko "Kirjallisuutta" -> "Lähteet"
\renewcommand{\refname}{Lähteet}  % article-tyyppisen

% Määritä bib-tiedoston nimi tähän (eli lahteet.bib ilman bib)
\bibliography{lahteet}

% ---------------------------------------------------------------------

\end{document}

% --------------------------------------------------------------------

\section{Introduction}

Communicating with each other using technologies, such as Bluetooth, is becoming ever more popular in the field. Both old and new emerging technologies enable us to create new ways of establishing communication between total strangers with similar interests. This thesis describes how these technologies can be used to create social interaction between strangers and therefore increase their well-being of people and their performance during sports.

Familiar strangers is a concept first introduced by the psychologist Stanley Milgram in 1972 in his essay \citep{milgram1992}. We often come across to the same strangers while doing sports, but do not interact with them. These people, that you have met frequently but never interacted with, are called familiar strangers. \citep{familiarStranger}. Familiar strangers as a concept isn't limited to sports, but targeting the research to people who have similar interests (sports) by definition makes monitoring of their behavior simpler. While social networking between strangers has been research before, this concept of social interaction between familiar strangers in sports, is new in the field.

Methods used in this paper to research this problems are:
\begin{itemize}
\item  Literature review.
\item Conducting interviews.
\item Creating a prototype application for research data.
\end{itemize}

This paper presents a prototype Android application that will log the times strangers passing by you. When you come across to a strangers enough times, the application will suggest communication with the stranger. With the prototype, you can view where and how many times you have encountered that person and what are they interested in. Interesting questions related to this prototype application are, whether users are willing to establish communication based on similar interest and similar real-life habits (sports routes and times) and also how much information users are willing to share to total strangers. Data gathered form this prototype application can later be used to verify assumptions about the users behavior and to learn new information. The prototype application takes privacy seriously and is quite conservative about sharing information. The level of privacy can later then be modified based on feedback from the users.

The interviews were composed from open-ended questions where the goal was more to find new information rather than just to validate previous assumptions. The interviews were extensive and performed only for a handful of possible end users of the application. No survey's were conducted for this thesis.

\section{Related work}

This section presents related works from two perspectives: the social interaction perspective and the perspective of doing sports. The design of the prototype application relies on results from both of these perspectives.

\subsection{Social interaction}

\cite{socialAdHoc} studied ad hoc social networking with a social networking system called TWIN. In a survey conducted after the study, the method for approaching unfamiliar persons was one of the highest rated features of the system. \cite{mobileMatchmaking} conducted a survey where 90\% of the participants stated that they would use regularly a service which would help introduce nearby strangers to each other. Serendipity, the application created for their research, is a mobile match-making system which alerts users when someone with similar interests comes into proximity. The reactions to the system have been overwhelmingly positive. These results imply that systems which allow people to interact with familiar strangers are in fact desired by users.

\subsection{Sports}

Meeting strangers is only one part of the assumed benefits of the prototype application. Previous research suggests that doing sports in a group or together with a friend results in increased performance. Therefore, the findings suggest that finding strangers with similar interests and a similar level of fitness to do sports with would result in a performance increase for the user. However, finding people to do sports with can be an daunting task especially for people who have just moved to a new city or a country. It is important to make finding strangers as easy as possible with the use of modern technology without compromising the privacy of the users.

One of the concerns of creating the application is that how frequently people doing sports actually meet familiar strangers. Setting the level of passing by's before allowing users to communicate with each other affects the whole user experience of the prototype application. Research by \cite{runningNavigation} showed that distance is the key thing what joggers are thinking about while running, not about using familiar routes. However, while routes change, joggers use familiar locations more than once. They usually leave out a part or add one based on their overall feeling. The fact, that joggers reuse locations increases the probability of running into familiar strangers along the way.

\section{Interviews}

This section describes the methods used for the interviews and the results gained from them. All interviews are anonymous and only basic demographic information about age was gathered from the participants.

\subsection{Method for the interviews}

The interviews consist multiple open-ended questions. The goal is to find out information about how people behave while doing sports. The main questions for the interview are:

\begin{itemize}
	\item Age?
	\item What sports are you doing?
	\item When are you planning to do sports next and what are you planning to do?
	\item What kind of sports goals do you have?
		\begin{itemize}
			\item What is keeping you from achieving them / what has helped you achieve them?
		\end{itemize}
	\item What do you carry with you while doing sports?
	\item Do you do it alone or with friends?
	\item Have you interacted with strangers while doing sports?
		\begin{itemize}
			\item If so, how?
		\end{itemize}
	\item How does your motivation differ while doing sports by yourself and in a group?
	\item Have you used any kind of tools to find people to do sports with?
	\item Do you use any sports tracking applications?
	\item What information about your sporting habits would you be willing to share with strangers that you come across often?
	\item Would you be interested in trying out a tool which would suggest messaging people that you come across often while doing sports?
\end{itemize}

The questions aren't meant to be strict, just a guideline for the discussion. If new interesting discussions emerge while interviewing, the point is to go forward with them without thinking too much about the guideline presented previously.

In total, X amount of people were interviewed for this study. X of them were male and Y of them were female.

\subsection{Results}



\section{Design}

First and foremost, the application's main goal is to allow people to connect with familiar strangers and find people to do sports with. Of course the application can be used for other purposes than just sports, but it is mainly designed for that challenge. It will be interesting to see whether it is used for just random encounteres instead of with a specific goal in mind. The designed process of meeting a familiar strangers and connecting with them divided into a few steps is:

\begin{itemize}
	\item Pass by the person enough times
	\item View information about the person and their interests when the application suggests communication.
	\item Hit them up with a message.
	\item View their real-life information after both agree to do it and start doing sports with them.
\end{itemize}

This prototype application is a mix of remote and social interaction among people assisted with techonology. Users are able to chat with familiar strangers remotely but real life proximity is required to start these conversations. The aim is to push the boundaries of how social interaction is done among strangers. Shared common interests are usually the starters for social interaction among strangers and the application tries to highlight these common interests and therefore, initialize social interaction among strangers.

\subsection{Passing by}

Using Bluetooth beacons, the application will log every person that passes by. Only one of the users have to have their mobile phone with them, since all the encounters are stored in the server. The hard part of this section is to select the amount of pass by's that initialize communication between the users. The initial default value is three times. This can easily be adjusted from the server later so no additional application releases need to be made. Three times most likely will be too low when the amount of users for the application grow. If a larger study is done later to validate the users behaviors, the amount should be upgraded. The users are also able to select the amount themselves before other users are able to approach them with messages. This serves all needs from the users.

The user is able to view how many people they have come across to from a list in the application. The list shows the user with an unique identification (not their real name) and also the times they have come across each other.

The logging can be turned on and off whenever the user feels like it. It will be interesting to see whether people will keep it on always, or just during the activities they want to perform with other people, such as sports.

\subsection{Collected information}

After the user has encountered a familiar strangers enough times, the application will suggest communication between both parties. The user will see the other persons public interest that they have filled while registering to the application and also display the locations on the map, where they have encountered each other. No other data is visible at this moment to the users. Based on the literature review of this thesis, displaying this kind of information publicly to the users isn't a problem at all. Most users would be willing to reveal even more information to the strangers based on the research done by X. However, the application deosn't have any additional information about the  interests of the person, so other information would only be users real name and pictures.

\subsection{Approaching}

The inital approach towards the other user can be done anonymously. It's possible to send messages to each other just to ask what they are interested in and see if both of them would be intersted in doing sports together. Research by X suggests that users are interested in getting to know other people, therefore it is possible to chat with strangers in this application. The chat is meant for users to verify similar interests and goals before getting to know each other better.

After this, or before this event, it is possible for the user to request real life information from the other person. Only after both parties agree to reveal information will anything be revealed. After that, it is possible for both of them to continue doing sports together and find meaningful things to do in life. A record of their communication is left on the app, and it will not log any pass by's from the other person anymore at this point. Until real-life information is accepted, the application will still log encounters from the person and view them in the list of encounters unless the user removes them manually from the list.

\section{Prototype implementation}

The created prototype is an Android application. The programming language for the application is Kotlin. Kotlin helped to reduce the amount of bugs during the creation process and proved to be a very useful selection for the application. In addition to Kotlin, anko; a view library created by JetBrains; was also used for the application. Using anko reduced the amount of time going to creating basic views, e.g. for the login and registering views.

Using the application requires users to get a Bluetooth beacon. The application uses the open  Eddystone protocol, so there is plenty of beacon models to choose from. Most popular Bluetooth beacon's at the moment of this thesis' creation are Apple's iBeacons and Estimote beacons.

\subsection{Architecture}

The application is divided into a client (the Android application) and a backend (server). The backend is built with Firebase, which is a tool for creating fast backends using JSON nothing but objects. The user authenticates to the backend, so that we get unique users that aren't locked into a single device in the application's lifetime.

The application uses Firebase to authenticate users. A combination of email and password is currently used for the authentication. However, it is easy to add third party login systems, such as Facebook and Google for the authentication.

The application uses a background service to run the logging in a separate thread. The application detects the "join" and "exit" events of Bluetooth beacons and based on those events, sends the server information about the encounters. This background service will stay running even though the user is not currently actively using the application. Monitoring the events for Bluetooth beacons drains extra battery from the phones.

Push notifications aren't at the moment handled by a real-time push notification server. The application polls, the Firebase server periodically and sees if there are any changes available. This will hopefully be improved in the future.

\subsection{Using the prototype}

The application is open sourced with the Apache 2.0 license. Therefore, anyone is free to take it into use, modify it or even make money out of it. A few steps are needed before the application can be taken into use by someone. First of all, the application uses Firebase as the backend service. Therefore, a Firebase account and an application created with it is required for using this application. The data structure for the JSON objects used in Firebase is described in the appliation's source code, so developing a custom backend for the application is entirely possible and quite fast to do.

\section{Discussion}

\subsection{Results of the study}

\subsection{Future work}

\section{Conclusion}

% --------------------------------------------------------------------
